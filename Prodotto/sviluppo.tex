\subsection{Strumenti utilizzati}
Per lo sviluppo di ENGaming, oltre a quanto già citato in \nameref{sec:sviluppoStrum}, ho utilizzato:

\begin{itemize}
    \item NodeJS, versione 18.16.0
    \item Npm, versione 9.5.1
    \item TypeScript, versione 4.9.5
    \item C++,versione 17
    \item AngularCLI, versione 15.2.8
    \item Electron, versione 25.0.1
    \item Git, versione 2.41.0
\end{itemize}
Inoltre, il codice versionato è stato caricato nella repository apposita, presente nello spazio \gls{aws} aziendale.
\subsection{Ambiente di sviluppo}
Per tenere una suddivisione chiara delle parti da sviluppare, ho impostato la cartella di lavoro secondo questa struttura:
\begin{itemize}
    \item src_angular \begin{itemize}
        \item app \begin{itemize}
            \item COMPONENTI
            \item <<servizi>>
            \item <<interfacce>>
        \end{itemize}
        \item assets \begin{itemize}
            \item games \begin{itemize}
                \item GIOCHI
            \end{itemize}
            \item previews \begin{itemize}
                \item <<immagini>>
            \end{itemize}
            \item games.json
            \item records.json
            \item <<altri file>>
        \end{itemize}
        \item index.html
        \item main.ts
        \item <<altri file>>
    \end{itemize}
    \item src_electron \begin{itemize}
        \item App.ts
        \item ENGaming.ts
    \end{itemize}
    \item src_module \begin{itemize}
        \item ES11LIB
        \item ES11LOADER
        \item LIBUSB
        \item <<altri file>>
    \end{itemize}
\end{itemize}
Dove:
\begin{itemize}
    \item COMPONENTI indica le cartelle di tutti i componenti, di cui la struttura si è precedentemente illustrata in \nameref{subsec:angular}.
    \item <<servizi>> indica i file dei servizi, che sono dotati solo di un file \Gls{tsc} per la definizione dei comportamenti.
    \item <<interfacce>> indica i file delle interfacce, dotati solo di un file Typescript per la definizione delle stesse.
    \item GIOCHI indica le cartelle dove sono presenti i giochi, con i relativi sorgenti.
    \item <<immagini>> indica le immagini utilizzate dall'applicativo.
    \item ES11LIB indica la cartella che contiene i driver per il funzionamento dell'ENSign11, ricevuti direttamente dall'azienda.
    \item ES11LOADER indica la cartella dove risiedono i file per la gestione dell'ENSign11, ai quali ho personalmente lavorato.
    \item LIBUSB indica la cartella dove risiede la libreria usata dai file contenuti in ES11LIB.
    \item <<altri file>> indica altri file, presenti nelle cartelle ma di minore interesse (ad esempio, file .gitignore)
\end{itemize}
Nello stesso livello delle cartelle principali, sono presenti anche i file di configurazione, in particolare i file \emph{package.json},\emph{angular.json}, \emph{tsconfig.json} e \emph{forge.config.js}.\\
Il progetto, durante il suo sviluppo, è stato versionato e caricato su una repository, presente in \gls{aws}.