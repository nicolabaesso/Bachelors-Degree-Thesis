% Acronyms
\newacronym[description={\glslink{apig}{Application Program Interface}}]
    {api}{API}{Application Program Interface}

\newacronym[description={\glslink{umlg}{Unified Modeling Language}}]
    {uml}{UML}{Unified Modeling Language}

\newacronym[description={\glslink{ideg}{Integrated Development Environment}}]
    {ide}{IDE}{Integrated Development Environment}

\newacronym[description={\glslink{ipcg}{Inter-Process Communication}}]
    {ipc}{IPC}{Inter-Process Communication}

% Glossary entries
\newglossaryentry{apig} {
    name=\glslink{apig}{API},
    text=Application Program Interface,
    sort=api,
    description={in informatica con il termine \emph{Application Programming Interface API} (ing. interfaccia di programmazione di un'applicazione) si indica ogni insieme di procedure disponibili al programmatore, di solito raggruppate a formare un set di strumenti specifici per l'espletamento di un determinato compito all'interno di un certo programma. La finalità è ottenere un'astrazione, di solito tra l'hardware e il programmatore o tra software a basso e quello ad alto livello semplificando così il lavoro di programmazione}
}

\newglossaryentry{umlg} {
    name=\glslink{umlg}{UML},
    text=UML,
    sort=uml,
    description={in ingegneria del software \emph{UML, Unified Modeling Language} (ing. linguaggio di modellazione unificato) è un linguaggio di modellazione e specifica basato sul paradigma object-oriented. L'\emph{UML} svolge un'importantissima funzione di ``lingua franca'' nella comunità della progettazione e programmazione a oggetti. Gran parte della letteratura di settore usa tale linguaggio per descrivere soluzioni analitiche e progettuali in modo sintetico e comprensibile a un vasto pubblico}
}

\newglossaryentry{ideg} {
    name=\glslink{ideg}{IDE},
    text=IDE,
    sort=ide,
    description={Un \emph{IDE}, o ambiente di sviluppo integrato, è un software progettato per la realizzazione di applicazioni che aggrega strumenti di sviluppo comuni in un'unica interfaccia utente grafica. In genere è costituito da un editor di testo che agevola la scrittura di codice software grazie a utili funzionalità come l'evidenziazione della sintassi con suggerimenti visivi, il completamento automatico specifico del linguaggio e l'individuazione di bug durante la scrittura. Inoltre, generalmente offre anche l'automazione della build locale e un debugger per risolvere problemi (conosciuti come \emph{bug}) nei programmi sviluppati}
}

\newglossaryentry{ipcg} {
    name=\glslink{ipcg}{IPC},
    text=IPC,
    sort=ipc,
    description={L'\emph{Inter-Process Communication} è un sistema di comunicazione tra processi offerto da \emph{Electron}. La comunicazione avviene dal processo \emph{main} al processo \emph{render} attraverso appositi canali, dove si possono inviare anche dati semplici. Le comunicazioni possono essere in forma sincrona o asincrona}
}

\newglossaryentry{angl} {
    name=Angular,
    text=Angular,
    sort=angular,
    description={\emph{Angular} (conosciuto anche come \emph{"Angular 2+"}) è un \emph{framework} per applicazioni web, gratuito e \emph{open source}, basato su \emph{TypeScript} e mantenuto sia dal \emph{Team} di \emph{Angular} in \emph{Google} sia dalla \emph{community}, da singoli e da altre aziende. Completa riscrittura dallo stesso team che ha sviluppato \emph{AngularJS}, \emph{Angular} è un \emph{framework} per applicazioni a pagina singola, utilizzato per la creazioni di applicazioni web veloci}
}

\newglossaryentry{tsc} {
    name=TypeScript,
    text=TypeScript,
    sort=typescript,
    description={\emph{TypeScript} è un linguaggio di programmazione \emph{open source} sviluppato da \emph{Microsoft}. Si tratta di un'estensione di \emph{JavaScript} che basa le sue caratteristiche su \emph{ECMAScript 6}. Il linguaggio estende la sintassi di \emph{JavaScript} in modo che qualunque programma scritto in \emph{JavaScript} sia anche in grado di funzionare con \emph{TypeScript} senza nessuna modifica. È stato progettato per lo sviluppo di grandi applicazioni e dev'essere compilato in \emph{JavaScript} per poter essere interpretato da qualunque web browser o app}
}

\newglossaryentry{elctr} {
    name=Electron,
    text=Electron,
    sort=electron,
    description={\emph{Electron} è un \emph{framework open source} gestito e ospitato da \emph{GitHub}. \emph{Electron} consente lo sviluppo della \emph{GUI} di applicazioni \emph{desktop} utilizzando tecnologie Web: combina il motore di \emph{rendering Chromium} e il \emph{runtime Node.js}. Le applicazioni \emph{Electron} sono composte da più processi: il processo \emph{"browser"} e diversi processi \emph{"renderer"}. Il processo \emph{browser} esegue la logica dell'applicazione e può quindi avviare più processi di \emph{rendering}, restituendo le finestre che appaiono sullo schermo di un utente processando \emph{HTML} e \emph{CSS}. Entrambi i processi \emph{browser} e \emph{renderer} possono essere eseguiti con l'integrazione di \emph{Node.js}}
}

\newglossaryentry{cpp} {
    name=C++,
    text=C++,
    sort=cpp,
    description={\emph{C++} è un linguaggio di programmazione orientato ad oggetti, cross-platform, che può essere utilizzato per lo sviluppo di applicazioni per alte prestazioni.
    Essendo uno dei linguaggi di programmazione più popolari, C++ può essere trovato nei sistemi operativi moderni, in interfacce grafice e sistemi \emph{embedded}.
    C++ nasce come estensione del linguaggio C, supportando la creazione di classi ed oggetti}
}

\newglossaryentry{napi} {
    name=Node-API,
    text=Node-API,
    sort=napi,
    description={\emph{Node-API} (conosciute anche come \emph{N-API}) sono delle API per creare Addons nativi. Sono indipendenti dal runtime Javascript ed è mantenuto come parte di Node.js. Il suo intento è di isolare gli addon dal runtime, permettendo agli stessi di girare per versioni successive di Node.js rispetto alla versione di compilazione, senza ricompilarli nuovamente}
}

\newglossaryentry{elctrForge} {
    name=Electron Forge,
    text=Electron Forge,
    sort=electron-forge,
    description={\emph{Electron Forge} è un tool all-in-one per pacchetizzare e distribuire applicazioni in Electron. Unisce molti pacchetti single-purpose per creare una pipeline completa che funziona out of the box, completa con code signing, installers, e pubblicazione degli artefatti}
}

\newglossaryentry{webassembly} {
    name=WebAssembly,
    text=WebAssembly,
    sort=webassembly,
    description={\emph{WebAssembly} è uno standard web che definisce un formato binario e un corrispondente formato testuale per la scrittura di codice eseguibile nelle pagine web. Ha lo scopo di abilitare l'esecuzione del codice quasi alla stessa velocità con cui esegue il codice macchina nativo. È stato progettato come integrazione di JavaScript per accelerare le prestazioni delle parti critiche delle applicazioni Web e in seguito per consentire lo sviluppo web in altri linguaggi oltre a JavaScript. È sviluppato dal World Wide Web Consortium (W3C) con ingegneri provenienti da Mozilla, Microsoft, Google e Apple. Viene eseguito in una sandbox nel browser Web dopo una fase di verifica formale. I programmi possono essere compilati da linguaggi di alto livello in moduli Wasm e caricati come librerie dalle applet JavaScript}
}

\newglossaryentry{displaylink} {
    name=DisplayLink,
    text=DisplayLink,
    sort=displaylink,
    description={\emph{DisplayLink} è un driver, sviluppato da \emph{Synaptics}, per l'utilizzo di device usb come monitor e docking stations. L'azienda offre i driver per Windows, MacOS, Android, ChromeOS e Ubuntu}
}

\newglossaryentry{latex} {
    name=LaTeX,
    text=LaTeX,
    sort=latex,
    description={\emph{LaTeX} è un sistema di composizione tipografica di alta qualità, che include funzionalità progettate per la produzione di documentazione tecnica e scientifica. LaTeX è lo standard de facto per la comunicazione e la pubblicazione di documenti scientifici. Inoltre, LaTeX è disponibile come software libero}
}