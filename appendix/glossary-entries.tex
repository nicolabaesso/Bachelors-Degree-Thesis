% Acronyms

\newacronym[description={\glslink{ideg}{Integrated Development Environment}}]
    {ide}{IDE}{Integrated Development Environment}

\newacronym[description={\glslink{ipcg}{Inter-Process Communication}}]
    {ipc}{IPC}{Inter-Process Communication}

\newacronym[description={\glslink{awsg}{Amazon Web Services}}]
    {aws}{AWS}{Amazon Web Services}

% Glossary entries
\newglossaryentry{apig} {
    name=Application Program Interface
    text=Application Program Interface,
    sort=api,
    description={in informatica con il termine \emph{Application Programming Interface, o API} (ing. interfaccia di programmazione di un'applicazione) si indica ogni insieme di procedure disponibili al programmatore, di solito raggruppate a formare un set di strumenti specifici per l'espletamento di un determinato compito all'interno di un certo programma. La finalità è ottenere un'astrazione, di solito tra l'hardware e il programmatore o tra software a basso e quello ad alto livello semplificando così il lavoro di programmazione}
}

\newglossaryentry{umlg} {
    name=Unified Modeling Language,
    text=UML,
    sort=uml,
    description={in ingegneria del software \emph{UML, Unified Modeling Language} (ing. linguaggio di modellazione unificato) è un linguaggio di modellazione e specifica basato sul paradigma object-oriented. L'\emph{UML} svolge un'importantissima funzione di ``lingua franca'' nella comunità della progettazione e programmazione a oggetti. Gran parte della letteratura di settore usa tale linguaggio per descrivere soluzioni analitiche e progettuali in modo sintetico e comprensibile a un vasto pubblico}
}

\newglossaryentry{ideg} {
    name=Integrated Development Environment,
    text=Integrated Development Environment,
    sort=ide,
    description={Un \emph{IDE}, Integrated Development Environment,, o ambiente di sviluppo integrato, è un software progettato per la realizzazione di applicazioni che aggrega strumenti di sviluppo comuni in un'unica interfaccia utente grafica. In genere è costituito da un editor di testo che agevola la scrittura di codice software grazie a utili funzionalità come l'evidenziazione della sintassi con suggerimenti visivi, il completamento automatico specifico del linguaggio e l'individuazione di bug durante la scrittura. Inoltre, generalmente offre anche l'automazione della build locale e un debugger per risolvere problemi (conosciuti come \emph{bug}) nei programmi sviluppati}
}

\newglossaryentry{ipcg} {
    name=Inter-Process Communication,
    text=Inter-Process Communication,
    sort=ipc,
    description={L'\emph{IPC, Inter-Process Communication}, è un sistema di comunicazione tra processi offerto da \emph{Electron}. La comunicazione avviene dal processo \emph{main} al processo \emph{render} attraverso appositi canali, dove si possono inviare anche dati semplici. Le comunicazioni possono essere in forma sincrona o asincrona}
}

\newglossaryentry{angl} {
    name=Angular,
    text=Angular,
    sort=angular,
    description={\emph{Angular} (conosciuto anche come \emph{"Angular 2+"}) è un \emph{framework} per applicazioni web, gratuito e \emph{open source}, basato su \emph{TypeScript} e mantenuto sia dal \emph{Team} di \emph{Angular} in \emph{Google} sia dalla \emph{community}, da singoli e da altre aziende. Completa riscrittura dallo stesso team che ha sviluppato \emph{AngularJS}, \emph{Angular} è un \emph{framework} per applicazioni a pagina singola, utilizzato per la creazioni di applicazioni web veloci}
}

\newglossaryentry{tsc} {
    name=TypeScript,
    text=TypeScript,
    sort=typescript,
    description={\emph{TypeScript} è un linguaggio di programmazione \emph{open source} sviluppato da \emph{Microsoft}. Si tratta di un'estensione di \emph{JavaScript} che basa le sue caratteristiche su \emph{ECMAScript 6}. Il linguaggio estende la sintassi di \emph{JavaScript} in modo che qualunque programma scritto in \emph{JavaScript} sia anche in grado di funzionare con \emph{TypeScript} senza nessuna modifica. È stato progettato per lo sviluppo di grandi applicazioni e dev'essere compilato in \emph{JavaScript} per poter essere interpretato da qualunque web browser o app}
}

\newglossaryentry{elctr} {
    name=Electron,
    text=Electron,
    sort=electron,
    description={\emph{Electron} è un \emph{framework open source} gestito e ospitato da \emph{GitHub}. \emph{Electron} consente lo sviluppo della \emph{GUI} di applicazioni \emph{desktop} utilizzando tecnologie Web: combina il motore di \emph{rendering Chromium} e il \emph{runtime Node.js}. Le applicazioni \emph{Electron} sono composte da più processi: il processo \emph{"browser"} e diversi processi \emph{"renderer"}. Il processo \emph{browser} esegue la logica dell'applicazione e può quindi avviare più processi di \emph{rendering}, restituendo le finestre che appaiono sullo schermo di un utente processando \emph{HTML} e \emph{CSS}. Entrambi i processi \emph{browser} e \emph{renderer} possono essere eseguiti con l'integrazione di \emph{Node.js}}
}

\newglossaryentry{cpp} {
    name=C++,
    text=C++,
    sort=cpp,
    description={\emph{C++} è un linguaggio di programmazione orientato ad oggetti, cross-platform, che può essere utilizzato per lo sviluppo di applicazioni per alte prestazioni.
    Essendo uno dei linguaggi di programmazione più popolari, C++ può essere trovato nei sistemi operativi moderni, in interfacce grafice e sistemi \emph{embedded}.
    C++ nasce come estensione del linguaggio C, supportando la creazione di classi ed oggetti}
}

\newglossaryentry{napi} {
    name=Node-API,
    text=Node-API,
    sort=napi,
    description={\emph{Node-API} (conosciute anche come \emph{N-API}) sono delle API per creare Addons nativi. Sono indipendenti dal runtime Javascript ed è mantenuto come parte di Node.js. Il suo intento è di isolare gli addon dal runtime, permettendo agli stessi di girare per versioni successive di Node.js rispetto alla versione di compilazione, senza ricompilarli nuovamente}
}

\newglossaryentry{elctrForge} {
    name=Electron Forge,
    text=Electron Forge,
    sort=electron-forge,
    description={\emph{Electron Forge} è un tool all-in-one per pacchetizzare e distribuire applicazioni in Electron. Unisce molti pacchetti single-purpose per creare una pipeline completa che funziona out of the box, completa con code signing, installers, e pubblicazione degli artefatti}
}

\newglossaryentry{webassembly} {
    name=WebAssembly,
    text=WebAssembly,
    sort=webassembly,
    description={\emph{WebAssembly} è uno standard web che definisce un formato binario e un corrispondente formato testuale per la scrittura di codice eseguibile nelle pagine web. Ha lo scopo di abilitare l'esecuzione del codice quasi alla stessa velocità con cui esegue il codice macchina nativo. È stato progettato come integrazione di JavaScript per accelerare le prestazioni delle parti critiche delle applicazioni Web e in seguito per consentire lo sviluppo web in altri linguaggi oltre a JavaScript. È sviluppato dal World Wide Web Consortium (W3C) con ingegneri provenienti da Mozilla, Microsoft, Google e Apple. Viene eseguito in una sandbox nel browser Web dopo una fase di verifica formale. I programmi possono essere compilati da linguaggi di alto livello in moduli Wasm e caricati come librerie dalle applet JavaScript}
}

\newglossaryentry{displaylink} {
    name=DisplayLink,
    text=DisplayLink,
    sort=displaylink,
    description={\emph{DisplayLink} è un driver, sviluppato da \emph{Synaptics}, per l'utilizzo di device usb come monitor e docking stations. L'azienda offre i driver per Windows, MacOS, Android, ChromeOS e Ubuntu}
}

\newglossaryentry{latex} {
    name=LaTeX,
    text=LaTeX,
    sort=latex,
    description={\emph{LaTeX} è un sistema di composizione tipografica di alta qualità, che include funzionalità progettate per la produzione di documentazione tecnica e scientifica. LaTeX è lo standard de facto per la comunicazione e la pubblicazione di documenti scientifici. Inoltre, LaTeX è disponibile come software libero}
}

\newglossaryentry{html} {
    name=HTML,
    text=HTML,
    sort=html,
    description={Sigla di \emph{Hyper text markup language}, HTML è un linguaggio di \emph{markup} utilizzato per la creazione di pagine web.
    Creato da \emph{Timothy J. Berners-Lee} all’inizio degli anni Novanta del secolo scorso, si basa su file di testo in cui alcuni marcatori, chiamati \emph{tags}, racchiusi tra simboli particolari, specificano proprietà e comportamenti di testi e immagini inseriti nella pagina.
    I tag HTML sono racchiusi tra parentesi angolari (< e >), scritti indifferentemente in lettere maiuscole o minuscole. 
    Una pagina \emph{HTML} è formata da una sezione di un’intestazione, ovvero il \emph{tag head}, e una di corpo, ovvero il \emph{tag body}, a loro volta racchiuse dai \emph{tag <html>} e \emph{</html>}: questi sono preceduti da una dichiarazione iniziale, definita con \emph{DOCTYPE}, che indica al \emph{browser} la versione \emph{HTML} utilizzata. L’evoluzione del linguaggio, da \emph{HTML} 1.0 a \emph{HTML} 4.01, ha permesso di inserire nella struttura delle pagine \emph{web} elementi sempre più complessi, come le tabelle, alcune funzioni di accessibilità per utenti diversamente abili, l’incorporamento di oggetti e la gestione dei moduli (detti \emph{form}, che rendono bidirezionale il flusso delle informazioni fra \emph{provider} del servizio e utente). 
    Un’evoluzione importante è l’\emph{HTML5} che, forzando la separazione fra struttura della pagina, gli stili e i contenuti, consente una più agevole realizzazione di applicazioni web, un miglior controllo dei contenuti multimediali e una maggior integrazione con i dispositivi mobili (per es. con funzioni di geolocalizzazione). 
    Inoltre, le funzioni di memorizzazione in locale dei contenuti web consentono l’utilizzo delle applicazioni \emph{HTML} anche in presenza di una connessione Internet poco stabile o performante}
}

\newglossaryentry{css} {
    name=CSS,
    text=CSS,
    sort=css,
    description={Sigla di \emph{Cascading Style Sheet}, ovvero foglio di stile a cascata, con CSS si indica un complesso di istruzioni o regole che serve a determinare l’aspetto grafico di un documento generato con un linguaggio a marcatori come \emph{HTML} o \emph{XML}. 
    Un foglio di stile permette di separare le istruzioni che riguardano caratteristiche come colori, caratteri, spazi e \emph{layout} dal contenuto del documento. 
    L’aspetto dei caratteri di un testo può essere regolato simultaneamente da due fogli differenti, uno creato dall’autore, l’altro dall’utente (per es. un ipovedente che desideri ingrandire il testo). 
    La sovrapposizione è possibile perché le regole funzionano a cascata, secondo gerarchie che fanno prevalere una sull’altra. 
    Viceversa, il medesimo foglio può formattare più documenti: agendo su di esso, si può definire in blocco il loro aspetto. 
    I \emph{browser} sono dotati di propri fogli "di default". 
    Le specifiche \emph{CSS} 1 sono state messe a punto per la prima volta nel 1996 dal \emph{W3C} (\emph{World Wide Web Consortium}); nel 1998 sono state emanate le \emph{CSS} 2 e nel 2004 le \emph{CSS} 2.1, con nuove funzionalità, un linguaggio ben supportato e la possibilità di creare fogli di stile separati per dispositivi portatili. 
    Le specifiche \emph{CSS} 3, oltre a presentare nuove funzionalità, correggono alcuni bug, ovvero errori di programma o di sistema, presenti nelle versioni precedenti}
}

\newglossaryentry{js} {
    name=Javascript,
    text=Javascript,
    sort=javascript,
    description={\emph{Javascript} è un linguaggio di scripting orientato agli oggetti, interpretato e debolmente tipizzato, comunemente utilizzato nelle pagine \emph{HTML} dei siti \emph{web} per arricchirle di funzionalità e aspetti dinamici. 
    Uno script \emph{Javascript} viene solitamente ospitato da un altro programma, nella maggior parte dei casi da un browser. 
    I browser più comuni incorporano un interprete \emph{Javascript} e quando viene visitata una pagina \emph{HTML} che contiene il codice di uno script \emph{Javascript}, quest'ultimo è eseguito dall’interprete contenuto nel browser stesso.  
    Se ospitato da un browser, \emph{Javascript} offre la possibilità di manipolarne gli elementi mediante gli oggetti del \emph{Document Object Model}, o \emph{DOM}: a titolo esemplificativo, sono oggetti del \emph{DOM} l’oggetto \emph{window}, che rappresenta una finestra del \emph{browser}, e l’oggetto \emph{document}, che rappresenta la pagina \emph{HTML} vera e propria. 
    \emph{Javascript} offre inoltre la possibilità di catturare alcuni eventi che si possono verificare durante la navigazione di una pagina \emph{HTML}, come il passaggio del \emph{mouse} su un’immagine o la pressione del tasto sinistro del \emph{mouse} su un’ancora, e abbinarli all’esecuzione di uno \emph{script Javascript}}
}