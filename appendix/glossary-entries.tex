% Acronyms
\newacronym[description={\glslink{apig}{Application Program Interface}}]
    {api}{API}{Application Program Interface}

\newacronym[description={\glslink{umlg}{Unified Modeling Language}}]
    {uml}{UML}{Unified Modeling Language}

\newacronym[description={\glslink{ideg}{Integrated Development Environment}}]
    {ide}{IDE}{Integrated Development Environment}

% Glossary entries
\newglossaryentry{apig} {
    name=\glslink{api}{API},
    text=Application Program Interface,
    sort=api,
    description={in informatica con il termine \emph{Application Programming Interface API} (ing. interfaccia di programmazione di un'applicazione) si indica ogni insieme di procedure disponibili al programmatore, di solito raggruppate a formare un set di strumenti specifici per l'espletamento di un determinato compito all'interno di un certo programma. La finalità è ottenere un'astrazione, di solito tra l'hardware e il programmatore o tra software a basso e quello ad alto livello semplificando così il lavoro di programmazione}
}

\newglossaryentry{umlg} {
    name=\glslink{uml}{UML},
    text=UML,
    sort=uml,
    description={in ingegneria del software \emph{UML, Unified Modeling Language} (ing. linguaggio di modellazione unificato) è un linguaggio di modellazione e specifica basato sul paradigma object-oriented. L'\emph{UML} svolge un'importantissima funzione di ``lingua franca'' nella comunità della progettazione e programmazione a oggetti. Gran parte della letteratura di settore usa tale linguaggio per descrivere soluzioni analitiche e progettuali in modo sintetico e comprensibile a un vasto pubblico}
}

\newglossaryentry{ideg} {
    name=\glslink{ide}{IDE},
    text=IDE,
    sort=ide,
    description={Un ambiente di sviluppo integrato (in inglese \emph{Integrated Development Environment} ovvero IDE), in informatica, è un software che, in fase di programmazione, supporta i programmatori nello sviluppo e debugging del codice sorgente di un programma.
    Spesso l'IDE aiuta lo sviluppatore segnalando errori di sintassi del codice direttamente in fase di scrittura, oltre a tutta una serie di strumenti e funzionalità di supporto alla fase stessa di sviluppo e debugging.}
}

\newglossaryentry{angl} {
    name=Angular,
    text=Angular,
    sort=angular,
    description={\emph{Angular} (conosciuto anche come \emph{"Angular 2+"}) è un \emph{framework} per applicazioni web, gratuito e \emph{open source}, basato su \emph{TypeScript} e mantenuto sia dal \emph{Team} di \emph{Angular} in \emph{Google} sia dalla \emph{community}, da singoli e da altre aziende. Completa riscrittura dallo stesso team che ha sviluppato \emph{AngularJS}, \emph{Angular} è un \emph{framework} per applicazioni a pagina singola, utilizzato per la creazioni di applicazioni web veloci.}
}

\newglossaryentry{tsc} {
    name=TypeScript,
    text=TypeScript,
    sort=typescript,
    description={\emph{TypeScript} è un linguaggio di programmazione \emph{open source} sviluppato da \emph{Microsoft}. Si tratta di un'estensione di \emph{JavaScript} che basa le sue caratteristiche su \emph{ECMAScript 6}. Il linguaggio estende la sintassi di \emph{JavaScript} in modo che qualunque programma scritto in \emph{JavaScript} sia anche in grado di funzionare con \emph{TypeScript} senza nessuna modifica. È stato progettato per lo sviluppo di grandi applicazioni e dev'essere compilato in \emph{JavaScript} per poter essere interpretato da qualunque web browser o app.}
}

\newglossaryentry{elctr} {
    name=Electron,
    text=Electron,
    sort=electron,
    description={\emph{Electron} è un \emph{framework open source} gestito e ospitato da \emph{GitHub}. \emph{Electron} consente lo sviluppo della \emph{GUI} di applicazioni \emph{desktop} utilizzando tecnologie Web: combina il motore di \emph{rendering Chromium} e il \emph{runtime Node.js}. Le applicazioni \emph{Electron} sono composte da più processi: il processo \emph{"browser"} e diversi processi \emph{"renderer"}. Il processo \emph{browser} esegue la logica dell'applicazione e può quindi avviare più processi di \emph{rendering}, restituendo le finestre che appaiono sullo schermo di un utente processando \emph{HTML} e \emph{CSS}. Entrambi i processi \emph{browser} e \emph{renderer} possono essere eseguiti con l'integrazione di \emph{Node.js}.}
}
