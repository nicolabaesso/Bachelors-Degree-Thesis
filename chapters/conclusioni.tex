\chapter{Conclusioni}
\label{cap:conclusioni}

\section{Raggiungimento degli obiettivi e analisi del prodotto}
Con questo tirocinio, ho soddisfatto il 95\% dei requisiti totali, soddisfando al 100\% i requisiti obbligatori, come riportato in tabella.
\begin{longtable}{|c|c|}
    \hline
    \thead{Codice}&\thead{Stato}\\
    \hline
    RF01&Soddisfatto\\
    \hline
    RF02&Soddisfatto\\
    \hline
    RF03&Soddisfatto\\
    \hline
    RF04&Soddisfatto\\
    \hline
    RF05&Soddisfatto\\
    \hline
    RF06&Soddisfatto\\
    \hline
    RF07&Soddisfatto\\
    \hline
    RF08&Soddisfatto\\
    \hline
    RF09&Soddisfatto\\
    \hline
    RF10&Soddisfatto\\
    \hline
    RF11&Soddisfatto\\
    \hline
    RF12&Soddisfatto\\
    \hline
    RF13&Soddisfatto\\
    \hline
    RF14&Soddisfatto\\
    \hline
    RF15&Soddisfatto\\
    \hline
    RF16&Soddisfatto\\
    \hline
    RF17&Soddisfatto\\
    \hline
    RF18&Soddisfatto\\
    \hline
    RF19&Soddisfatto\\
    \hline
    RF20&Soddisfatto\\
    \hline
    RF21&Soddisfatto\\
    \hline
    RF22&Soddisfatto\\
    \hline
    RF23&Soddisfatto\\
    \hline
    RF24&Soddisfatto\\
    \hline
    RF25&Soddisfatto\\
    \hline
    RF26&Soddisfatto\\
    \hline
    RF27&Soddisfatto\\
    \hline
    RF28&Soddisfatto\\
    \hline
    RF29&Soddisfatto\\
    \hline
    RF30&Soddisfatto\\
    \hline
    RF31&Soddisfatto\\
    \hline
    RF32&Soddisfatto\\
    \hline
    RF33&Soddisfatto\\
    \hline
    RF34&Non Soddisfatto\\
    \hline
    RF35&Non Soddisfatto\\
    \hline
    RF36&Soddisfatto
    \hline
    \caption{Tabella di soddisfacimento dei requisiti}
\end{longtable}
Non ho potuto soddisfare i requisiti opzionali (R34 e R35) poiché il tempo necessario per il loro completamento era superiore al tempo rimastomi.\\
Quanto sviluppato corrisponde dunque ai requisiti analizzati e al comportamento aspettato.\\
In particolare, come anche già visto in \nameref{subsec:panoramicaProdotto}, ENGaming permette di:
\begin{itemize}
    \item visualizzare le informazioni relativa ad un gioco.
    \item avviare il gioco ed interagirci attraverso l'interfaccia presente.
    \item mettere in pausa, riprendere ed uscire dal gioco.
    \item salvare i record effettuati.
    \item visualizzare i record effettuati.
\end{itemize}
\section{Conoscenze acquisite}
In primis, ho imparato molto sulla creazione di elementi web tramite Angular. Venendo da una conoscenza "base", ovvero da solo HTML, CSS e JS senza l'uso di framework, ho trovato interessanti i contenuti che Angular stesso propone.\\
Ho sicuramente apprezzato l'utilizzo di Typescript, che con la tipizzazione è stato concettualmente semplice da lavorarci, avendo un background con linguaggi tipizzati (come Java e C++).\\
Inoltre, l'utilizzo delle Node-API per poter eseguire codice C++ in ambito web è stato molto interessante, sopratutto a livello comunicativo, venedo utilizzate tecnologie ideate per ambiti diversi.
\section{Valutazione personale}
Questo tirocinio ha sicuramente aiutato la mia crescita sia professionale che personale.\\
Gli strumenti che ho utilizzato si sono rivelati adeguati allo scopo, e ciò ha reso nettamente più facile l'intero processo di sviluppo.\\
Le fasi, che inizialmente vedevo come limitate temporalmente, si sono dimostrate adatte e mi hanno permesso di non avere ritardi e completare quanto poi definito.\\
Effettuare analisi e progettazione mi hanno permesso di non avere intoppi durante lo sviluppo. Inoltre, mi hanno aiutato a migliorare rispetto a quanto visto durante il percorso accademico.\\
Le tecnologie utilizzate sono state per me molto interessanti, a prescindere dalla conoscenza pregressa che avevo.\\
L'ambiente di lavoro è stato positivo e stimolante, e ha sicuramente aiutato nella realizzazione di questo progetto.\\
Nel complesso, sono soddisfatto di come è andato il tirocinio, delle conoscenze apprese e del risultato ottenuto. 