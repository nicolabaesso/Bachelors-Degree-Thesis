\chapter{Introduzione}
\label{cap:introduzione}

%Introduzione al contesto applicativo.\\

%\noindent Esempio di utilizzo di un termine nel glossario \\
%\gls{api}. \\

%\noindent Esempio di citazione in linea \\
%%\cite{site:agile-manifesto}. \\

%\noindent Esempio di citazione nel pie' di pagina \\
%citazione\footcite{womak:lean-thinking} \\

\section{Euronovate Group}

Euronovate Group è un gruppo multinazionale con oltre 150 dipendenti, leader nel mercato
nell'implementazione e commercializzazione di soluzioni innovative per la trasformazione digitale,
Certification Authority eIDAS compliant, e produttrice di oltre 50 prodotti proprietari HW e
SW.
Ha sede centrale a Mendrisio (Svizzera) e controllate con presenza diretta in 4 paesi:
\begin{itemize}
    \item Italia (Padova, Reggio Emilia, Milano)
    \item Spagna (Barcellona, Madrid, Bilbao)
    \item Romania (Bucarest)
    \item Cina (Shanghai)
\end{itemize}
Euronovate Group si suddivide in:
\begin{itemize}
    \item Euronovate SA
    \item eSignWorld
    \item Vintegris
\end{itemize}
\subsection{Euronovate SA}
Fondata nel 2012 e con sede a Lugano (CH), è una società svizzera innovativa, leader in soluzioni
di Digital Trasformation con approccio end-to-end, combinando soluzioni software, hardware
e servizi di consulenza.
L'obiettivo principale è aiutare ogni tipo di azienda ad eliminare tutti i processi e i documenti
cartacei passando completamente al digitale, garantendo la stessa validità legale.
\subsection{eSignWorld}
Società che opera el settore della consulenza IT, processi e sistemi avanzati di firme elettroniche,
fornitura, esercizio e manutenzione di sistemi informativi hardware e software, specializzata nel
campo della produzione a filiera corta di tecnologia grafometrica. In particolare, eSignWorld
fornisce soluzioni personalizzate e proprietarie nel campo della dematerializzazione documentale
e dell'Information Communication Technology, garantendo la possibilità di visualizzare,
elaborare e firmare elettronicamente qualsiasi tipo di documento. eSignworld offre soluzioni
paperless con utilizzo di firma elettronica semplice e firma elettronica avanzata, un sistema di
composizione e successiva conservazione di documenti in formato elettronico, nonché di firma dei
documenti, attraverso un innovativo dispositivo di firma.
\subsection{Vintegris}
Víntegris progetta, implementa e gestisce infrastrutture di sicurezza per istituzioni finanziarie
e grandi società. Fornisce solide soluzioni su misura per ogni esigenza di business, integrando
tecnologie ad alte prestazioni e soddisfacendo le esigenze di ogni azienda. La società definisce e
realizza progetti ad alto valore aggiunto per la protezione delle informazioni, la sicurezza web, il
controllo e la gestione degli accessi. Vengono utilizzate tecnologie tra le più robuste sul mercato
(Docker, Kubernetes, AWS, HSM), integrandole nell'ambiente aziendale di ciascun cliente. L'es-
sere consulenti e integratori, pone Víntegris in una posizione privilegiata nello sviluppo di nuovi
prodotti: la conoscenza delle esigenze e delle criticità delle grandi aziende in materia di sicurezza
delle informazioni, guida Víntegris verso la progettazione di prodotti che coprono il divario tra le
soluzioni di sicurezza dei grandi produttori e le reali esigenze aziendali. In questo modo, l'azienda
è in grado di anticipare le esigenze di mercato in segmenti critici come la gestione, il controllo e
gli audit di certificati digitali e autenticazione dell'utente. Tutti i consulenti di Víntegris hanno
una vasta esperienza nel campo della tecnologia dell'informazione e sono esperti nella progetta-
zione, implementazione e gestione di infrastrutture di sicurezza delle informazioni. Inoltre, la
maggior parte dei consulenti è in possesso di certificazioni riconosciute a livello internazionale,
come CISA, CISSP e CISM, che garantiscono la loro esperienza e conoscenza nell'ambito della
sicurezza delle informazioni.

\section{ENSign 11}

L'ENSign 11 \footcite{site:ensign11}  è un tablet versatile per firme grafometriche. Con le sue differenti applicazioni, è lo strumento perfetto per la digitalizzazione delle firme a mano.\\
L'ENSign è dotato delle seguenti caratteristiche:
\begin{itemize}
    \item dotato di pannello multi-tocco con isolamento nativo dello schermo
    \item connessione diretta ad un computer
    \item altamente compatto, con grande stabilità, alti livelli di sicurezza e design elegante
\end{itemize}

Features
Unique signature pad with Multi-touch panel with native segregation of screen.
Directly connected to a computer, it offers a highly compact device with great stability, high levels of security and an elegant design.
Fulfils the highest levels of national and international safety standard of the Finite Element Analysis (FEA).
The pad captures a person's handwritten signature and the biometric parameters such as pressure, acceleration, speed, rhythm, and movements in the air, embedding the biometric data in an electronic PDF document.
Every ENSIGN 11 pad uses a proprietary system with separate types of encryption for each form of transaction. The encryption engine prevents access to circuits and any attempt of sniffing sensitive data.
ENSIGN 11 versatility gives the option to turn your biometric signature pad into a second screen with multi-touch functionalities just by connecting the tablet to the computer USB port and installing the required ENSIGN 11 trayApp to have access to all the multimedia functions.

This technology becomes the perfect tool for many industries due to its multiple applications and its potential to create the right remote work environment as two people can be interacting in a document in real time.

With ENSIGN 11 Multi-touch screen it is possible to create, write, draw, and display customizable multimedia contents (tutorials, video clips, slide shows, web pages, online lessons) in real-time, which can be automatically shared through videoconference or in a business presentation.



\section{L'idea}

Introduzione all'idea dello stage.

\section{Organizzazione del testo}

\begin{description}
    \item[{\hyperref[cap:processi-metodologie]{Il secondo capitolo}}] descrive ...
    
    \item[{\hyperref[cap:descrizione-stage]{Il terzo capitolo}}] approfondisce ...
    
    \item[{\hyperref[cap:analisi-requisiti]{Il quarto capitolo}}] approfondisce ...
    
    \item[{\hyperref[cap:progettazione-codifica]{Il quinto capitolo}}] approfondisce ...
    
    \item[{\hyperref[cap:verifica-validazione]{Il sesto capitolo}}] approfondisce ...
    
    \item[{\hyperref[cap:conclusioni]{Nel settimo capitolo}}] descrive ...
\end{description}

Riguardo la stesura del testo, relativamente al documento sono state adottate le seguenti convenzioni tipografiche:
\begin{itemize}
	\item gli acronimi, le abbreviazioni e i termini ambigui o di uso non comune menzionati vengono definiti nel glossario, situato alla fine del presente documento;
	\item per la prima occorrenza dei termini riportati nel glossario viene utilizzata la seguente nomenclatura: \emph{parola}\glsfirstoccur;
	\item i termini in lingua straniera o facenti parti del gergo tecnico sono evidenziati con il carattere \emph{corsivo}.
\end{itemize}
