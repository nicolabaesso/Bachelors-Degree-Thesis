\chapter{Introduzione}
\label{cap:introduzione}

%Introduzione al contesto applicativo.\\

%\noindent Esempio di utilizzo di un termine nel glossario \\
%\gls{api}. \\

%\noindent Esempio di citazione in linea \\
%%\cite{site:agile-manifesto}. \\

%\noindent Esempio di citazione nel pie' di pagina \\
%citazione\footcite{womak:lean-thinking} \\

In questa sezione vengono descritte l'azienda ospitante, il progetto utilizzato per il tirocinio ed il \emph{device} utilizzato per lo stesso.

\section{Euronovate Group}

\emph{Euronovate Group} è un gruppo multinazionale con oltre 150 dipendenti, leader nel mercato
nell'implementazione e commercializzazione di soluzioni innovative per la trasformazione digitale,
Certification Authority eIDAS compliant, e produttrice di oltre 50 prodotti proprietari HW e
SW.\\
Ha sede centrale a Mendrisio (Svizzera) e controllate con presenza diretta in 4 paesi:
\begin{itemize}
    \item Italia (Padova, Reggio Emilia, Milano)
    \item Spagna (Barcellona, Madrid, Bilbao)
    \item Romania (Bucarest)
    \item Cina (Shanghai)
\end{itemize}
Euronovate Group inoltre si suddivide in:
\begin{itemize}
    \item Euronovate SA
    \item eSignWorld
    \item Vintegris
\end{itemize}
\subsection{Euronovate SA}
Fondata nel 2012 e con sede a Lugano (CH), è una società svizzera innovativa, leader in soluzioni
di Digital Trasformation con approccio end-to-end, combinando soluzioni software, hardware
e servizi di consulenza.\\
L'obiettivo principale è aiutare ogni tipo di azienda ad eliminare tutti i processi e i documenti
cartacei passando completamente al digitale, garantendo la stessa validità legale.
\subsection{eSignWorld}
Società che opera el settore della consulenza IT, processi e sistemi avanzati di firme elettroniche,
fornitura, esercizio e manutenzione di sistemi informativi hardware e software, specializzata nel
campo della produzione a filiera corta di tecnologia grafometrica.\\ In particolare, eSignWorld
fornisce soluzioni personalizzate e proprietarie nel campo della dematerializzazione documentale
e dell'Information Communication Technology, garantendo la possibilità di visualizzare,
elaborare e firmare elettronicamente qualsiasi tipo di documento.\\ eSignworld offre soluzioni
paperless con utilizzo di firma elettronica semplice e firma elettronica avanzata, un sistema di
composizione e successiva conservazione di documenti in formato elettronico, nonché di firma dei
documenti, attraverso un innovativo dispositivo di firma.
\subsection{Vintegris}
Víntegris progetta, implementa e gestisce infrastrutture di sicurezza per istituzioni finanziarie
e grandi società.\\ Fornisce solide soluzioni su misura per ogni esigenza di business, integrando
tecnologie ad alte prestazioni e soddisfacendo le esigenze di ogni azienda.\\ La società definisce e
realizza progetti ad alto valore aggiunto per la protezione delle informazioni, la sicurezza web, il
controllo e la gestione degli accessi.\\ Vengono utilizzate tecnologie tra le più robuste sul mercato
(Docker, Kubernetes, AWS, HSM), integrandole nell'ambiente aziendale di ciascun cliente.\\ L'es-
sere consulenti e integratori, pone Víntegris in una posizione privilegiata nello sviluppo di nuovi
prodotti: la conoscenza delle esigenze e delle criticità delle grandi aziende in materia di sicurezza
delle informazioni, guida Víntegris verso la progettazione di prodotti che coprono il divario tra le
soluzioni di sicurezza dei grandi produttori e le reali esigenze aziendali.\\ In questo modo, l'azienda
è in grado di anticipare le esigenze di mercato in segmenti critici come la gestione, il controllo e
gli audit di certificati digitali e autenticazione dell'utente.\\ Tutti i consulenti di Víntegris hanno
una vasta esperienza nel campo della tecnologia dell'informazione e sono esperti nella progetta-
zione, implementazione e gestione di infrastrutture di sicurezza delle informazioni.\\ Inoltre, la
maggior parte dei consulenti è in possesso di certificazioni riconosciute a livello internazionale,
come CISA, CISSP e CISM, che garantiscono la loro esperienza e conoscenza nell'ambito della
sicurezza delle informazioni.

\section{Il progetto: ENGaming}

Uno dei principali mercati dove opera Euronovate è il mercato dei device di firma grafometrica
dotati di un monitor da 10 pollici multi-touch.\\ Questi dispositivi USB possono essere utilizzati
per molteplici scopi oltre all'inserimento di una firma in un documento, nel tempo l'azienda ha visto
realizzare sistemi di presentazioni slideshow, digital signage, questionari di valutazione e form
per l'aggiornamento delle anagrafiche.\\ Oggi Euronovate vuole spingersi ulteriormente nel mondo dei
multimedia e utilizzare il device di firma ENSign 11 per l'intrattenimento videoludico.\\
L'obiettivo dello stage è apprendere le nozioni dello sviluppo di applicazioni desktop e di integrazioni di dispositivi fisici, lavorando in team e seguendo una pianificazione
settimanale delle attività.\\ 
Ulteriori feature di prodotto possono essere proposte dallo stagista e saranno valutate dal tutor.

\section{ENSign 11}

L'ENSign 11 \footcite{site:ensign11}  è un tablet versatile per firme grafometriche. Con le sue differenti applicazioni, è lo strumento perfetto per la digitalizzazione delle firme a mano.\\
L'ENSign è dotato delle seguenti caratteristiche:
\begin{itemize}
    \item pannello multi-tocco con isolamento nativo dello schermo
    \item connessione diretta ad un computer
    \item altamente compatto, con grande stabilità, alti livelli di sicurezza e design elegante
    \item cattura di parametri biometrici come pressione, accelerazione, velocità, ritmo e movimento in aria
    \item sistema proprietario di criptazione per ogni tipo di transazione
\end{itemize}
Inoltre, la versatilità dell'ENSign 11 permette di trasformarlo in un secondo schermo, semplicemente collegandolo tramite USB ad un computer e installando l'apposita applicazione ENSIGN 11 trayApp, in modo da avere tutte le funzionalità multimediali.\\
Con lo schermo multi-touch di ENSign 11 è possibile creare, scrivere, disegnare e mostrare contenuti multimediali personalizzati in tempo reale, che possono essere automaticamente condivisi attraverso videoconferenze o in presentazioni d'affari.

\section{Strumenti di comunicazione}
\subsection{Gmail}
Gmail è un servizio gratuito non-libero di posta elettronica supportato da pubblicità fornito da Google. 
È possibile accedervi via web o tramite applicazioni che usano i protocolli POP3, IMAP o le API Google.
È stato pubblicato il 1º aprile 2004 e solo il 7 luglio 2009, dopo oltre 5 anni di permanenza nello status di beta pubblica, è stato reso definitivo.
Dispone di 15 GB di spazio gratuito (condiviso con tutti gli altri servizi offerti da Google all'utente), ulteriormente aumentabili con pacchetti a pagamento. 
La versione principale webmail è realizzata in AJAX, è tuttavia disponibile una versione HTML che non necessita di JavaScript.
\subsection{Google Meet}
Google Meet, per uso desktop, è un prodotto software di tipo SaaS cioè un servizio usufruibile mediante browser: a differenza di altre piattaforme di virtual meeting non richiede alcun client installato. 
Invece, per i dispositivi mobili (Android 4, 4.1, 5, 5.1) si può usare un'app (Google Meet)
Sui dispositivi mobili Android (6 +) e iOS non è necessaria l'app, ma basta avere Gmail e cliccare la telecamera in basso. 
Come per l'intero ecosistema Google, il servizio è integrato con Gmail e Google Calendar per programmare e notificare le call.
Gli utenti hanno bisogno di un account Google per avviare chiamate e, come gli utenti di G Suite, chiunque disponga di un account Google è in grado di avviare una chiamata Meet da Gmail.
Le chiamate gratuite per riunioni non hanno limiti di tempo, ma saranno limitate a 60 minuti a partire da settembre 2020. 
Per motivi di sicurezza, gli host possono negare l'ingresso e rimuovere gli utenti durante una chiamata.
A partire da aprile 2020, Google prevede di implementare un filtro audio con cancellazione del rumore e una modalità per la scarsa illuminazione.[23][24]

Google Meet utilizza protocolli proprietari per la transcodifica di video, audio e dati. Tuttavia, Google ha stretto una partnership con la società Pexip per fornire l'interoperabilità tra Google Meet e le apparecchiature e il software per conferenze basati su SIP/H.323.[25] Poiché Meet viene eseguito in un browser e non richiede un'app o un'estensione, dovrebbe presentare meno vulnerabilità di sicurezza rispetto ai servizi di videoconferenza che richiedono un'app desktop.[26][27]

Google Meet consente agli utenti in abbonamento di pubblicare su YouTube le proprie riunioni in modalità live streaming .[28]
\subsection{Telegram}
Telegram è un servizio di messaggistica istantanea e broadcasting basato su cloud ed erogato senza fini di lucro dalla società Telegram LLC, una società a responsabilità limitata con sede a Dubai,[6] fondata dall'imprenditore russo Pavel Durov.[7] I client ufficiali di Telegram sono distribuiti come software libero per Android, Linux, iOS, MacOS, Windows.[8]

Caratteristiche di Telegram sono la possibilità di scambiare messaggi di testo tra due utenti o tra gruppi, effettuare chiamate vocali e videochiamate cifrate punto-punto, scambiare messaggi vocali, videomessaggi, fotografie, video, sticker e file di qualsiasi tipo fino a 2 GB.[9][10] Attraverso i canali è anche possibile la trasmissione in diretta di audio/video e testo verso i membri che si uniscono. È inoltre possibile programmare l’orario di invio di un messaggio audio/testo; impostare un timer per l'autodistruzione dei messaggi che permette l'eliminazione automatica del messaggio una volta visualizzato dal destinatario, così come cancellare messaggi anche per il destinatario, e modificarne il testo dopo l'invio
\section{Linguaggi utilizzati}

\section{Strumenti di sviluppo}

\section{Organizzazione del testo}

\begin{description}
    \item[{\hyperref[cap:processi-metodologie]{Il secondo capitolo}}] descrive ...
    
    \item[{\hyperref[cap:descrizione-stage]{Il terzo capitolo}}] approfondisce ...
    
    \item[{\hyperref[cap:analisi-requisiti]{Il quarto capitolo}}] approfondisce ...
    
    \item[{\hyperref[cap:progettazione-codifica]{Il quinto capitolo}}] approfondisce ...
    
    \item[{\hyperref[cap:verifica-validazione]{Il sesto capitolo}}] approfondisce ...
    
    \item[{\hyperref[cap:conclusioni]{Nel settimo capitolo}}] descrive ...
\end{description}

%Riguardo la stesura del testo, relativamente al documento sono state adottate le seguenti convenzioni tipografiche:
%\begin{itemize}
%	\item gli acronimi, le abbreviazioni e i termini ambigui o di uso non comune menzionati vengono definiti nel glossario, situato alla fine del presente documento;
%	\item per la prima occorrenza dei termini riportati nel glossario viene utilizzata la seguente nomenclatura: \emph{parola}\glsfirstoccur;
%	\item i termini in lingua straniera o facenti parti del gergo tecnico sono evidenziati con il carattere \emph{corsivo}.
%\end{itemize}
