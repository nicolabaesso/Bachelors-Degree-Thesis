\chapter{Verifica e validazione}
\label{cap:verifica-validazione}

Verifica e validazione sono due parti fondamentali dello sviluppo di un software.\\
Il processo di verifica permette di controllare che quanto è prodotto sia effettivamente congruo con i requisiti, attraverso test automatici o manuali.\\
Il processo di validazione permette la convalida di quanto prodotto, secondo quanto precedentemente verificato.\\\\
Per la fase di verifica, per questo progetto, non sono ricorso all'utilizzo di test automatizzati, poiché i casi d'uso e i comportamenti che l'applicativo esegue sono definiti e limitati in quantità.\\
Dunque, ho effettuato la verifica sulle singole parti in maniera manuale, attraverso il debug tramite gli strumenti da me scelti, che hanno reso la stessa poco costosa a livello temporale.\\\\
La fase di validazione è stata fatta assieme al tutor, in due momenti distinti.\\
Infatti, la prima validazione è stata effettuata ancor prima della fase di Analisi, che ha decretato quanto prodotto durante la fase di Formazione Personale come Proof of Concept\footnote{Con Proof of Concept si intende una realizzazione incompleta o abbozzata di un progetto, allo scopo di dimostrarne la fattibilità.}.\\
La seconda (e "vera") validazione ha coinvolto l'applicativo vero e proprio, collaudandolo e accettandolo come prodotto finale.
