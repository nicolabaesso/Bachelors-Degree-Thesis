% Variables
\newcommand{\myName}{Nicola Baesso}
\newcommand{\myTitle}{Trasformare i device di firma in console portatili tramite l'ENGaming }
\newcommand{\myDegree}{Tesi di laurea}
\newcommand{\myUni}{Università degli Studi di Padova}
\newcommand{\myFaculty}{Corso di Laurea in Informatica}
\newcommand{\myDepartment}{Dipartimento di Matematica ``Tullio Levi-Civita''}
\newcommand{\profTitle}{Prof.}
\newcommand{\myProf}{Enrico Palazzi}
\newcommand{\myTutor}{Matteo Gnoato}
\newcommand{\myLocation}{Sant'Angelo di Piove}
\newcommand{\myAA}{2022-2023}
\newcommand{\myTime}{Settembre 2023}

% PDF/A filecontents
\RequirePackage{filecontents}
\begin{filecontents*}{\jobname.xmpdata}
  \Title{Trasformare i device di firma in console portatili tramite l'ENGaming}
  \Author{Nicola Baesso}
  \Language{it-IT}
  \Subject{Il presente documento espone il lavoro svolto dal laureando Nicola Baesso, presso l’azienda ESignWorld S.R.L., durante il tirocinio della durata di circa trecento ore. Il tirocinio prevedeva inanzitutto lo sviluppo di un applicativo desktop da utilizzare in un device di firma, unendo tecnologie front-end e back-end. In particolare, era richiesto lo sviluppo dell’applicazione, utilizzando i framework Angular ed Electron. Inoltre, volendo recuperare i dati dal device di firma, l’applicazione necessitava l’utilizzo di moduli compilati per NodeJS, e scritti in C++. Lo scopo di tale progetto era dimostrare che il device utilizzato non si limitasse alla semplice firma, ma che potesse essere un ottimo strumento multimediale, in particolare un ottimo strumento per l’utilizzo di videogames.}
  \Keywords{ENGaming\sep Device di firma\sep Console}
\end{filecontents*}

% Page format settings
% see: http://wwwcdf.pd.infn.it/AppuntiLinux/a2547.htm
\setlength{\parindent}{14pt}    % first row indentation
\setlength{\parskip}{0pt}       % paragraphs spacing

% Acronyms
\newacronym[description={\glslink{apig}{Application Program Interface}}]
    {api}{API}{Application Program Interface}

\newacronym[description={\glslink{umlg}{Unified Modeling Language}}]
    {uml}{UML}{Unified Modeling Language}

\newacronym[description={\glslink{ide}{Integrated Development Environment}}]
    {ide}{IDE}{Integrated Development Environment}

\newacronym[description={\glslink{ipcg}{Inter-Process Communication}}]
    {ipc}{IPC}{Inter-Process Communication}

% Glossary entries
\newglossaryentry{apig} {
    name=\glslink{api}{API},
    text=Application Program Interface,
    sort=api,
    description={in informatica con il termine \emph{Application Programming Interface API} (ing. interfaccia di programmazione di un'applicazione) si indica ogni insieme di procedure disponibili al programmatore, di solito raggruppate a formare un set di strumenti specifici per l'espletamento di un determinato compito all'interno di un certo programma. La finalità è ottenere un'astrazione, di solito tra l'hardware e il programmatore o tra software a basso e quello ad alto livello semplificando così il lavoro di programmazione}
}

\newglossaryentry{umlg} {
    name=\glslink{uml}{UML},
    text=UML,
    sort=uml,
    description={in ingegneria del software \emph{UML, Unified Modeling Language} (ing. linguaggio di modellazione unificato) è un linguaggio di modellazione e specifica basato sul paradigma object-oriented. L'\emph{UML} svolge un'importantissima funzione di ``lingua franca'' nella comunità della progettazione e programmazione a oggetti. Gran parte della letteratura di settore usa tale linguaggio per descrivere soluzioni analitiche e progettuali in modo sintetico e comprensibile a un vasto pubblico}
}

\newglossaryentry{ideg} {
    name=\glslink{ide}{IDE},
    text=IDE,
    sort=ide,
    description={Un ambiente di sviluppo integrato (in inglese \emph{Integrated Development Environment} ovvero IDE), in informatica, è un software che, in fase di programmazione, supporta i programmatori nello sviluppo e debugging del codice sorgente di un programma.
    Spesso l'IDE aiuta lo sviluppatore segnalando errori di sintassi del codice direttamente in fase di scrittura, oltre a tutta una serie di strumenti e funzionalità di supporto alla fase stessa di sviluppo e debugging.}
}

\newglossaryentry{ipcg} {
    name=\glslink{ipc}{IPC},
    text=IPC,
    sort=ipc,
    description={L'\emph{Inter-Process Communication} è un sistema di comunicazione tra processi offerto da \emph{Electron}. La comunicazione avviene dal processo \emph{main} al processo \emph{render} attraverso appositi canali, dove si possono inviare anche dati semplici. Le comunicazioni possono essere in forma sincrona o asincrona.}
}

\newglossaryentry{angl} {
    name=Angular,
    text=Angular,
    sort=angular,
    description={\emph{Angular} (conosciuto anche come \emph{"Angular 2+"}) è un \emph{framework} per applicazioni web, gratuito e \emph{open source}, basato su \emph{TypeScript} e mantenuto sia dal \emph{Team} di \emph{Angular} in \emph{Google} sia dalla \emph{community}, da singoli e da altre aziende. Completa riscrittura dallo stesso team che ha sviluppato \emph{AngularJS}, \emph{Angular} è un \emph{framework} per applicazioni a pagina singola, utilizzato per la creazioni di applicazioni web veloci.}
}

\newglossaryentry{tsc} {
    name=TypeScript,
    text=TypeScript,
    sort=typescript,
    description={\emph{TypeScript} è un linguaggio di programmazione \emph{open source} sviluppato da \emph{Microsoft}. Si tratta di un'estensione di \emph{JavaScript} che basa le sue caratteristiche su \emph{ECMAScript 6}. Il linguaggio estende la sintassi di \emph{JavaScript} in modo che qualunque programma scritto in \emph{JavaScript} sia anche in grado di funzionare con \emph{TypeScript} senza nessuna modifica. È stato progettato per lo sviluppo di grandi applicazioni e dev'essere compilato in \emph{JavaScript} per poter essere interpretato da qualunque web browser o app.}
}

\newglossaryentry{elctr} {
    name=Electron,
    text=Electron,
    sort=electron,
    description={\emph{Electron} è un \emph{framework open source} gestito e ospitato da \emph{GitHub}. \emph{Electron} consente lo sviluppo della \emph{GUI} di applicazioni \emph{desktop} utilizzando tecnologie Web: combina il motore di \emph{rendering Chromium} e il \emph{runtime Node.js}. Le applicazioni \emph{Electron} sono composte da più processi: il processo \emph{"browser"} e diversi processi \emph{"renderer"}. Il processo \emph{browser} esegue la logica dell'applicazione e può quindi avviare più processi di \emph{rendering}, restituendo le finestre che appaiono sullo schermo di un utente processando \emph{HTML} e \emph{CSS}. Entrambi i processi \emph{browser} e \emph{renderer} possono essere eseguiti con l'integrazione di \emph{Node.js}.}
}

\newglossaryentry{cpp} {
    name=C++,
    text=C++,
    sort=cpp,
    description={\emph{C++} è un linguaggio di programmazione general purpose sviluppato in origine da \emph{Bjarne Stroustrup} nei \emph{Bell Labs} nel 1983 come evoluzione del linguaggio \emph{C}, inserendo la programmazione orientata agli oggetti. Col tempo ha avuto notevoli evoluzioni, come l'introduzione dell'astrazione rispetto al tipo.
    C++ è il fratello minore di C ed è ancora oggi uno dei linguaggi di programmazione più utilizzati accompagnato da C, C#, Python, Java e JavaScript.}
}

\newglossaryentry{napi} {
    name=Node-API,
    text=Node-API,
    sort=napi,
    description={\emph{Node-API} (conosciute anche come \emph{N-API}) sono delle API per creare Addons nativi. Sono indipendenti dal runtime Javascript ed è mantenuto come parte di Node.js. Il suo intento è di isolare gli addon dal runtime, permettendo agli stessi di girare per versioni successive di Node.js rispetto alla versione di compilazione, senza ricompilarli nuovamente.}
}

\newglossaryentry{elctrForge} {
    name=Electron Forge,
    text=Electron Forge,
    sort=electron-forge,
    description={\emph{Electron Forge} è un tool all-in-one per pacchetizzare e distribuire applicazioni in Electron. Unisce molti pacchetti single-purpose per creare una pipeline completa che funziona out of the box, completa con code signing, installers, e pubblicazione degli artefatti.}
}

\newglossaryentry{webassembly} {
    name=WebAssembly,
    text=WebAssembly,
    sort=webassembly,
    description={\emph{WebAssembly} è uno standard web che definisce un formato binario e un corrispondente formato testuale per la scrittura di codice eseguibile nelle pagine web. Ha lo scopo di abilitare l'esecuzione del codice quasi alla stessa velocità con cui esegue il codice macchina nativo. È stato progettato come integrazione di JavaScript per accelerare le prestazioni delle parti critiche delle applicazioni Web e in seguito per consentire lo sviluppo web in altri linguaggi oltre a JavaScript. È sviluppato dal World Wide Web Consortium (W3C) con ingegneri provenienti da Mozilla, Microsoft, Google e Apple. Viene eseguito in una sandbox nel browser Web dopo una fase di verifica formale. I programmi possono essere compilati da linguaggi di alto livello in moduli Wasm e caricati come librerie dalle applet JavaScript.}
}
\makeglossaries

\bibliography{appendix/bibliography}

\defbibheading{bibliography} {
    \cleardoublepage
    \phantomsection
    \addcontentsline{toc}{chapter}{\bibname}
    \chapter*{\bibname\markboth{\bibname}{\bibname}}
}

% Spacing between entries
\setlength\bibitemsep{1.5\itemsep}

\DeclareBibliographyCategory{opere}
\DeclareBibliographyCategory{web}

\addtocategory{opere}{womak:lean-thinking}
\addtocategory{web}{site:agile-manifesto}

\defbibheading{opere}{\section*{Riferimenti bibliografici}}
\defbibheading{web}{\section*{Siti Web consultati}}


\captionsetup{
    tableposition=top,
    figureposition=bottom,
    font=small,
    format=hang,
    labelfont=bf
}

% Images path
\graphicspath{{images/}}

\hypersetup{
    %hyperfootnotes=false,
    %pdfpagelabels,
    colorlinks=true,
    linktocpage=true,
    pdfstartpage=1,
    pdfstartview=,
    breaklinks=true,
    pdfpagemode=UseNone,
    pageanchor=true,
    pdfpagemode=UseOutlines,
    plainpages=false,
    bookmarksnumbered,
    bookmarksopen=true,
    bookmarksopenlevel=1,
    hypertexnames=true,
    pdfhighlight=/O,
    %nesting=true,
    %frenchlinks,
    urlcolor=webbrown,
    linkcolor=RoyalBlue,
    citecolor=webgreen
    %pagecolor=RoyalBlue,
}

% Delete all links and show them in black
\if \isprintable 1
    \hypersetup{draft}
\fi

% Itemize symbols
%\renewcommand{\labelitemi}{$\ast$}
%\renewcommand{\labelitemi}{$\bullet$}
%\renewcommand{\labelitemii}{$\cdot$}
%\renewcommand{\labelitemiii}{$\diamond$}
%\renewcommand{\labelitemiv}{$\ast$}

% Listings setup
\lstset{
    language=[LaTeX]Tex,%C++,
    keywordstyle=\color{RoyalBlue}, %\bfseries,
    basicstyle=\small\ttfamily,
    %identifierstyle=\color{NavyBlue},
    commentstyle=\color{Green}\ttfamily,
    stringstyle=\rmfamily,
    numbers=none, %left,%
    numberstyle=\scriptsize, %\tiny
    stepnumber=5,
    numbersep=8pt,
    showstringspaces=false,
    breaklines=true,
    frameround=ftff,
    frame=single
}

\definecolor{webgreen}{rgb}{0,.5,0}
\definecolor{webbrown}{rgb}{.6,0,0}

% \omiss produces '[...]'
\newcommand{\omissis}{[\dots\negthinspace]}

% Hyphenation rules
\hyphenation {
    ma-cro-istru-zio-ne
    gi-ral-din
}

\newcommand{\sectionname}{sezione}
\addto\captionsitalian{\renewcommand{\figurename}{Figura}
                       \renewcommand{\tablename}{Tabella}}

\newcommand{\glsfirstoccur}{\ap{{[g]}}}

\newcommand{\intro}[1]{\emph{\textsf{#1}}}

% Risks environment
\newcounter{riskcounter}                % define a counter
\setcounter{riskcounter}{0}             % set the counter to some initial value

%%%% Parameters
% #1: Title
\newenvironment{risk}[1]{
    \refstepcounter{riskcounter}        % increment counter
    \par \noindent                      % start new paragraph
    \textbf{\arabic{riskcounter}. #1}   % display the title before the content of the environment is displayed
}{
    \par\medskip
}

\newcommand{\riskname}{Rischio}

\newcommand{\riskdescription}[1]{\textbf{\\Descrizione:} #1.}

\newcommand{\risksolution}[1]{\textbf{\\Soluzione:} #1.}

% Use case environment
\newcounter{usecasecounter}             % define a counter
\setcounter{usecasecounter}{0}          % set the counter to some initial value

%%%% Parameters
% #1: ID
% #2: Nome
\newenvironment{usecase}[2]{
    \renewcommand{\theusecasecounter}{\usecasename #1}  % this is where the display of
                                                        % the counter is overwritten/modified
    \refstepcounter{usecasecounter}             % increment counter
    \vspace{10pt}
    \par \noindent                              % start new paragraph
    {\large \textbf{\usecasename #1: #2}}       % display the title before the
                                                % content of the environment is displayed
    \medskip
}{
    \medskip
}

\newcommand{\usecasename}{UC}

\newcommand{\usecaseactors}[1]{\textbf{\\Attori Principali:} #1. \vspace{4pt}}
\newcommand{\usecasepre}[1]{\textbf{\\Precondizioni:} #1. \vspace{4pt}}
\newcommand{\usecasedesc}[1]{\textbf{\\Descrizione:} #1. \vspace{4pt}}
\newcommand{\usecasepost}[1]{\textbf{\\Postcondizioni:} #1. \vspace{4pt}}
\newcommand{\usecasealt}[1]{\textbf{\\Scenario Alternativo:} #1. \vspace{4pt}}

% Namespace description environment
\newenvironment{namespacedesc}{
    \vspace{10pt}
    \par \noindent  % start new paragraph
    \begin{description}
}{
    \end{description}
    \medskip
}

\newcommand{\classdesc}[2]{\item[\textbf{#1:}] #2}

\newcommand\minput[1]{%
    \input{#1}%
    \ifhmode\ifnum\lastnodetype=11 \unskip\fi\fi
}
